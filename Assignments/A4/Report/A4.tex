\documentclass[final,titlepage,onecolumn]{article}
\usepackage[margin=1.0in]{geometry}
\usepackage{amsmath}
\usepackage{graphicx}
\usepackage{amssymb}
\usepackage[final]{hyperref}
\usepackage[section]{placeins}
\usepackage{titlepic}
\usepackage{forloop}
\usepackage{listings}
\usepackage{xcolor}
\usepackage[export]{adjustbox}
\usepackage{url}
\usepackage{titlepic}
\usepackage{tikz}
\usetikzlibrary{shapes.geometric, arrows}
\usepackage{siunitx}


\definecolor{mGreen}{rgb}{0,0.6,0}
\definecolor{mGray}{rgb}{0.5,0.5,0.5}
\definecolor{mPurple}{rgb}{0.58,0,0.82}
\definecolor{backgroundColour}{rgb}{0.95,0.95,0.92}

\lstdefinestyle{Cpp}{language=C++,
	backgroundcolor=\color{backgroundColour},   
	commentstyle=\color{mGreen},
	keywordstyle=\color{magenta},
	numberstyle=\tiny\color{mGray},
	stringstyle=\color{mPurple},
	basicstyle=\footnotesize,
	breakatwhitespace=false,         
	breaklines=true,                 
	captionpos=b,                    
	keepspaces=true,                 
	numbers=left,                    
	numbersep=5pt,                  
	showspaces=false,                
	showstringspaces=false,
	showtabs=false,                  
	tabsize=4,
	basicstyle=\tiny\ttfamily}

\lstdefinestyle{None}{	
	backgroundcolor=\color{backgroundColour},   
	commentstyle=\color{black},
	keywordstyle=\color{black},
	numberstyle=\tiny\color{black},
	stringstyle=\color{black},
	basicstyle=\footnotesize,
	basicstyle=\tiny\ttfamily,
	breaklines=true
}

\lstloadlanguages{Matlab}%
\lstdefinestyle{Matlab}
{
	language=Matlab,
	backgroundcolor=\color{backgroundColour},   
	commentstyle=\color{mGreen},
	keywordstyle=\color{magenta},
	numberstyle=\tiny\color{mGray},
	stringstyle=\color{mPurple},
	basicstyle=\footnotesize,
	breakatwhitespace=false,         
	breaklines=true,                 
	captionpos=b,                    
	keepspaces=true,                 
	numbers=left,                    
	numbersep=5pt,                  
	showspaces=false,                
	showstringspaces=false,
	showtabs=false,                  
	tabsize=4,
	basicstyle=\tiny\ttfamily}


%opening
\title{ECE 7440 - Wavefield Imaging and Inversion\\
Assignment \# 4\\
Born Iterative Methods}
\author{Max Hughson}
\date{\today}
%\titlepic{\includegraphics[width=0.8\textwidth]{./figs/title.jpg}}



\newcommand{\twonorm}[1]{\left|#1\right|_2}
\newcommand{\infnorm}[1]{\left|#1\right|_\infty}
\newcommand{\hnorm}[1]{\left|#1\right|_h}
\newcommand{\rond}[2]{\frac{\partial#2}{\partial#1}}
\newcommand{\rrond}[2]{\frac{\partial^2#2}{\partial#1^2}}
\newcommand{\rondcross}[3]{\frac{\partial^2#3}{\partial #1 \partial #2}}
\newcommand{\curl}[1]{\nabla\times\left( #1\right)}
\newcommand{\divg}[1]{\nabla\cdot\left( #1\right)}
\newcommand{\grad}[1]{\nabla \left(#1\right)}
\newcommand{\curlm}{\begin{bmatrix}
	0&-\partial_z&\partial_y\\
	\partial_z&0&-\partial_x\\
	-\partial_y&\partial_x&0\end{bmatrix}}
\newcommand{\ccurlm}{\begin{bmatrix}
		-\partial_z^2-\partial_y^2&\partial_x\partial_y&\partial_x\partial_z\\
		\partial_x\partial_y&-\partial_x^2-\partial_z^2&\partial_y\partial_z\\
		\partial_x\partial_z&\partial_y\partial_z&-\partial_x^2-\partial_y^2
\end{bmatrix}}
\newcommand{\delbydel}[2]{\frac{\partial #1}{\partial #2}}
\newcommand{\delsqbyddel}[3]{\frac{\partial^2 #1}{\partial #2 \partial #3}}
\newcommand{\threevec}[1]{\begin{bmatrix}#1_x\\#1_y\\#1_z\end{bmatrix}}

\begin{document}

\maketitle

\begin{center}
\Large{\href{https://github.com/hughsonm/WFI/tree/master/Assignments/A4}{\textcolor{blue}{$ \rightarrow $Link to my code$ \leftarrow $}}}	
\end{center}

\normalsize

\FloatBarrier
\section{Important Notes About My Solvers}
\begin{itemize}
	\item They support multi-frequency
	\item They support frequency-dependent, transmitter-dependent receiver masks
	\begin{itemize}
		\item Achieved by sticking a restriction matrix on the left of both sides of data equationmathw
		\item $ S_{11} $ was ignored
		\item A random selection of other receivers were ignored
	\end{itemize}
	\item The Born iterative method uses truncated singular-value decomposition
	\item The distorted Born iterative method uses truncated conjugate-gradient least-squares
	\item All inversions presented here are on noiseless data
	\item My stopping condition was based on number of iterations
\end{itemize}

\FloatBarrier
\section{Synthetic Data Generation}
\begin{figure}[h]
	\centering
	\includegraphics[width=0.75\linewidth]{Figs/20191218T173302/fwd_fields}
	\caption[fwd_fields]{Low-Contrast Forward Solve}
	\label{fig:fwdfields}
\end{figure}



\FloatBarrier
\section{Born Iterative Method}
The process for the Born iterative method is to make the Born approximation in order to calculate a contrast estimate, then use the contrast estimate to improve the total field approximation. Square brackets denote diagonal matrices and double-underscores denote vertical concatenation of multi-frequency, multi-transmitter data.

\FloatBarrier
\subsection{Method Outline}
\begin{align*}
\text{Set } \underline{\chi}^{(0)}&\\
\forall i \in \left[1,N\right]:&\\
\text{For each tx, }&\underline{u}^{T(i)}_{DOM} = \left(I-k_0^2G_{0,DOM}\left[\underline{\chi}^{i-1}\right]\right)\underline{u}^{I}_{DOM}\\
&\underline{\chi}^{i} = \left(\underline{\underline{k_0^2G_{0,DATA}\left[u^{T,(i)}\right]}}\right)^{\dagger}\underline{\underline{u}}^{S,MEAS}
\end{align*}

\FloatBarrier
\subsection{Low-Contrast Target}
\begin{figure}[h]
	\centering
	\includegraphics[width=0.7\linewidth]{Figs/20191219T120041/bim_contrast}
	\caption[bim_contrast]{Contrast from Born Iterative Method}
	\label{fig:bimcontrast}
\end{figure}
\begin{figure}[h]
	\centering
	\includegraphics[width=0.7\linewidth]{Figs/20191219T120041/bim_fields}
	\caption[bim_fields]{Fields from Born Iterative Method}
	\label{fig:bimfields}
\end{figure}
\begin{figure}[h]
	\centering
	\includegraphics[width=0.7\linewidth]{Figs/20191219T120041/bim_Fs}
	\caption[bim_fs]{Cost Functional for Born Iterative Method}
	\label{fig:bimfs}
\end{figure}


\FloatBarrier
\subsection{Extreme Reconstruction}
I tried playing around with the contrast of the outer square in my domain. With the distorted Born iterative method, I was able to image a maximum contrast of roughly 1.0.
\begin{figure}[h]
	\centering
	\includegraphics[width=0.7\linewidth]{Figs/20191219T131756/bim_contrast}
	\caption[bim_extreme]{Extreme Reconstruction with Born Iterative Method}
	\label{fig:bimextreme}
\end{figure}




\FloatBarrier
\section{Distorted Born Iterative Method}
The process for the distorted Born iterative method is to make the Born approximation in order to calculate a contrast estimate, then use the contrast estimate to improve the total field approximation. Furthermore, the total fields generated by the contrast estimate are used to build a distorted Green's function operator. The contrast that is recovered from the data equation is a contrast with respect to the previous contrast, not a contrast with respect to the homogeneous background.
Use the following terms to refer to different media:
\begin{itemize}
	\item[\textbf{Incident}] refers to the fields and Green's functions in a homogeneous medium
	\item[\textbf{Background}] refers to the fields and Green's functions in a non-homogeneous medium
	\item[\textbf{Total}] refers to the fields and Green's functions in another non-homogeneous medium which represents the actual target and actual measurements
\end{itemize}


\FloatBarrier
\subsection{Method Outline}
\begin{align*}
	\underline{u}^{\text{BKG}}_{\text{DOM}} &= \left(I-k_0^2G^{\text{INC}}_{\text{DOM}}\left[\underline{\chi}^{\text{BKG}}\right]\right)^{-1}\underline{u}^{\text{INC}}_{\text{DOM}}\\
	\underline{u}^{\text{BKG}}_{\text{DAT}} &= \underline{u}^{\text{INC}}_{\text{DAT}} + k_0^2G^{\text{INC}}_{\text{DAT}}\left[u^{\text{BKG}}_{\text{DOM}}\right]\underline{\chi}^{\text{BKG}}\\
	G^{\text{BKG}}_{\text{DAT}}	&\text{ Built by placing point-sources at receiver locations and calculating fields.}\\
	\underline{\Delta\varepsilon} &=\left(M_Sk_0^2G^{\text{BKG}}_{\text{DAT}}\left[\underline{u}^{\text{BKG}}_{\text{DOM}}\right]\right)^\dagger M_S\left(\underline{u}^{\text{TOT}}_{\text{DAT}}-\underline{u}^{\text{BKG}}_{\text{DAT}}\right)\\
	\underline{\chi}^{\text{BKG}} & \gets \underline{\chi}^{\text{BKG}} + \underline{\Delta\varepsilon}\\
	&\text{Repeat until stopping condition is met.}
\end{align*}

\FloatBarrier
\subsection{Low-Contrast Target}
\begin{figure}[h]
	\centering
	\includegraphics[width=0.7\linewidth]{Figs/20191219T120041/dbim_contrast}
	\caption[dbim_contrast]{Contrast from Distorted Born Iterative Method}
	\label{fig:dbimcontrast}
\end{figure}
\begin{figure}[h]
	\centering
	\includegraphics[width=0.7\linewidth]{Figs/20191219T120041/dbim_fields}
	\caption[dbim_fields]{Fields from Distorted Born Iterative Method}
	\label{fig:dbimfields}
\end{figure}
\begin{figure}[h]
	\centering
	\includegraphics[width=0.7\linewidth]{Figs/20191219T120041/dbim_Fs}
	\caption[dbim_fs]{Cost Functional for Distorted Born Iterative Method}
	\label{fig:dbimfs}
\end{figure}
\FloatBarrier
\subsection{Extreme Reconstruction}
I tried playing around with the contrast of the outer square in my domain. With the distorted Born iterative method, I was able to image a maximum contrast of roughly 2.5.
\begin{figure}[h]
	\centering
	\includegraphics[width=0.7\linewidth]{Figs/20191219T130054/dbim_contrast}
	\caption[dbim_extreme]{Extreme Reconstruction with Distorted Born Iterative Method}
	\label{fig:dbimextreme}
\end{figure}


\end{document}
